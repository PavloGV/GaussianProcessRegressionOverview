\documentclass[professionalfont,10pt]{beamer}
\usetheme{Pavlo}				% My custom theme

\usepackage{newtxtext} % For Times New Roman font
\usepackage[utf8]{inputenc}
\usepackage[T1]{fontenc}
%Information to be included in the title page:
\title{Introduction to Gaussian Process Regression as Ordinary Kriging}
\subtitle{A Brief Overview Presentation}
\author{Pavlo Vlastos}
%\date{August 25th, 2020}
\institute{Autonomous Systems Laboratory \\ The University of California at Santa Cruz}
\titlegraphic{\includegraphics[width=2.5cm]{./../images/ucsc-seal.jpg}}
\usepackage{caption}
\usepackage[font=footnotesize]{caption}
\captionsetup{font=footnotesize}
\usepackage{subcaption}
\usepackage{multicol}
\usepackage{bibentry}
\usepackage{hyperref}
\usepackage{lgreek}
\usepackage{algorithm2e,algorithmic,float}
\setbeamertemplate{bibliography item}[text]
\AtBeginSection

\begin{document}
	
	\begin{frame}
		\titlepage
	\end{frame}

	\begin{frame}{Table of Contents}
		\frametitle{Table of Contents}
		\begin{minipage}[t]{0.2\linewidth}\vspace{-0.5cm}
			\vfill%
		\end{minipage}
			\hfill%
		\begin{minipage}[t]{0.86\linewidth}\vspace{-0.5cm}
			%			\begin{multicols}{2} % USE IF THE TABLE OF CONTENTS IS TOO LONG
				\tableofcontents
			%			\end{multicols}
		\end{minipage}
		\vfill%
	\end{frame}
	
	%%%%%%%%%%%%%%%%%%%%%%%%%%%%%%%%%%%%%%%%%%%%%%%%%%%%%%%%%%%%%%%%%%%%%%%%%%%%%%%%%%%%%%%%%%%%%%%
	% Section: Introduction
	%%%%%%%%%%%%%%%%%%%%%%%%%%%%%%%%%%%%%%%%%%%%%%%%%%%%%%%%%%%%%%%%%%%%%%%%%%%%%%%%%%%%%%%%%%%%%%%
	\section{Introduction}
	\begin{frame}[t]{Introduction}
		\frametitle{Introduction}
		\hskip-0.75cm
		\begin{minipage}[t]{0.2\linewidth}\vspace{-0.5cm}
			\tiny\tableofcontents[currentsection,currentsubsection,hideothersubsections,subsectionstyle=show/shaded]
		\end{minipage}
		\hfill%
		\begin{minipage}[t]{0.86\linewidth}\vspace{-0.5cm}
			
		\end{minipage}
	\vfill%
	\end{frame}

	%%%%%%%%%%%%%%%%%%%%%%%%%%%%%%%%%%%%%%%%%%%%%%%%%%%%%%%%%%%%%%%%%%%%%%%%%%%%%%%%%%%%%%%%%%%%%%%
	% Section: Building a Field
	%%%%%%%%%%%%%%%%%%%%%%%%%%%%%%%%%%%%%%%%%%%%%%%%%%%%%%%%%%%%%%%%%%%%%%%%%%%%%%%%%%%%%%%%%%%%%%%
	\section{Building a Field}
	\begin{frame}[t]{Building a Field}
		\frametitle{Building a Field}
		\framesubtitle{Gaussian Random Field (GRF)}
		\hskip-0.75cm
		\begin{minipage}[t]{0.2\linewidth}\vspace{-0.5cm}
			\tiny\tableofcontents[currentsection,currentsubsection,hideothersubsections,subsectionstyle=show/shaded]
		\end{minipage}
		\hfill%
		\begin{minipage}[t]{0.86\linewidth}\vspace{-0.5cm}
			\begin{figure}[t]
				\centering
				\captionsetup{width=0.9\textwidth}
				\includegraphics[width=0.6\textwidth]{../images/GRFbeforeConv.png}
				\caption{A Gaussian Random Field (GRF) with values separated by a desired spacing parameter, or resolution}
			\end{figure}
		\end{minipage}
		\vfill%
	\end{frame}

	\begin{frame}[t]{Building a Field}
		\frametitle{Building a Field}
		\framesubtitle{Gaussian Random Field (GRF)}
		\hskip-0.75cm
		\begin{minipage}[t]{0.2\linewidth}\vspace{-0.5cm}
			\tiny\tableofcontents[currentsection,currentsubsection,hideothersubsections,subsectionstyle=show/shaded]
		\end{minipage}
		\hfill%
		\begin{minipage}[t]{0.86\linewidth}\vspace{-0.5cm}
			\begin{flushleft}
				$d_{xy}$ : smallest distance separating any two adjacent random values 
				
				$\underline{\text{x}}$ : $1 \times n$ Vector of x-coordinates, $\left[\begin{array}{cccc}x_0 & x_1 & \dots & x_{n-1}\end{array}\right]$,\\
				$ x_i~=~jd_{xy}, j~\in~\mathbb{Z}$
				
				$\underline{\text{y}}$ : $m \times 1$ Vector of y-coordinates, $\left[\begin{array}{cccc}y_0 & y_1 & \dots & y_{n-1}\end{array}\right]$,\\
				$ y_i~=~jd_{xy}, j~\in~\mathbb{Z}$

				
				$\underline{\text{Z}}$ : $m \times n$ Matrix of z-axis field values 
				
				$\underline{\text{K}}$ : $k \times k$ Matrix representing the kernel for convolution
	
				$\underline{\text{C}}$ : $N \times N$ Matrix representing the covariance matrix
			\end{flushleft}
		\end{minipage}
		\vfill%
	\end{frame}
	
	\subsection{Gaussian Kernel}
	\begin{frame}[t]{Building a Field}
		\frametitle{Building a Field}
		\framesubtitle{Gaussian Kernel}
		\hskip-0.75cm
		\begin{minipage}[t]{0.2\linewidth}\vspace{-0.5cm}
			\tiny\tableofcontents[currentsection,currentsubsection,hideothersubsections,subsectionstyle=show/shaded]
		\end{minipage}
		\hfill%
		\begin{minipage}[t]{0.86\linewidth}\vspace{-0.5cm}
			\begin{flushleft}
				\begin{itemize}
					\item A kernel is required to smooth the GRF to better describe some phenomenon. A 2D Gaussian function can be chosen for a Gaussian kernel,
					\begin{equation}
						g(x, y) = \frac{1}{2\pi\sigma^2}e^{-\Big(\frac{(x-\mu_x)^2 + (y-\mu_y)^2}{2\sigma^2}\Big)}
					\end{equation}
					where $\mu_x$ and $\mu_y$ are the respective means of the distributions, but usually are zero, and $\sigma$ is the standard deviation. \\
					\item To form the Gaussian kernel:
					\begin{equation}
					\underline{\text{K}} = \frac{1}{\sum_{i=0}^{k-1}\sum_{j=0}^{k-1} \text{K}_{i, j}}\Bigg(\sum_{i=0}^{k-1}\sum_{j=0}^{k-1} \text{K}_{i, j} = g(j, i)\Bigg)
					\end{equation}
					where $\text{K}_{i,j}$ is an element of $\underline{\text{K}}$
				\end{itemize}
			\end{flushleft}
		\end{minipage}
		\vfill%	
	\end{frame}

	\begin{frame}[t]{Building a Field}
		\frametitle{Building a Field}
		\framesubtitle{Gaussian Kernel}
		\hskip-0.75cm
		\begin{minipage}[t]{0.2\linewidth}\vspace{-0.5cm}
			\tiny\tableofcontents[currentsection,currentsubsection,hideothersubsections,subsectionstyle=show/shaded]
		\end{minipage}
		\hfill%
		\begin{minipage}[t]{0.86\linewidth}\vspace{-0.5cm}
			\begin{flushleft}
				A pseudo code implementation for the Gaussian kernel may assume the following form:
				\begin{algorithm}[H]
					\begin{algorithmic}[1]
						\STATE \text{sum} = 0
						\FOR{$i=0$ to $k-1$}
						\FOR{$j=0$ to $k-1$}
						\STATE $\underline{K}_{i, j} = g(j, i)$
						\STATE $\text{sum} += \underline{K}_{i, j}$
						\ENDFOR
						\ENDFOR
						\STATE $\underline{K}_{i, j} /= \text{sum}$
					\end{algorithmic}
					\label{alg:seq}
				\end{algorithm}
			\end{flushleft}
		\end{minipage}
		\vfill%
	\end{frame}
	
	\begin{frame}[t]{Building a Field}
		\frametitle{Building a Field}
		\framesubtitle{Gaussian Kernel}
		\hskip-0.75cm
		\begin{minipage}[t]{0.2\linewidth}\vspace{-0.5cm}
			\tiny\tableofcontents[currentsection,currentsubsection,hideothersubsections,subsectionstyle=show/shaded]
		\end{minipage}
		\hfill%
		\begin{minipage}[t]{0.86\linewidth}\vspace{-0.5cm}
			\begin{figure}[t]
				\centering
				\captionsetup{width=0.9\textwidth}
				\includegraphics[width=0.7\textwidth]{../images/GaussianKernel.png}
				\caption{A 2D Gaussian kernel, $\mu_x = \mu_y = 0, \sigma = 8$}
			\end{figure}
		\end{minipage}
		\vfill%
	\end{frame}
	
	\subsection{2D Convolution}
	\begin{frame}[t]{Building a Field}
	\frametitle{Building a Field}
	\framesubtitle{2D Convolution}
	\hskip-0.75cm
	\begin{minipage}[t]{0.2\linewidth}\vspace{-0.5cm}
		\tiny\tableofcontents[currentsection,currentsubsection,hideothersubsections,subsectionstyle=show/shaded]
	\end{minipage}
	\hfill%
	\begin{minipage}[t]{0.86\linewidth}\vspace{-0.5cm}
		\begin{flushleft}
			The Gaussian kernel can be convolved with the GRF
			\begin{equation}
				\underline{\text{Z}}_{\text{new}} = \sum_{p=0}^{m_{\underline{\text{Z}}}-1}\sum_{q=0}^{n_{\underline{\text{Z}}}-1}\bigg(
				\underbrace{\underline{\text{Z}}_{\text{new},p,q} = \sum_{u=0}^{m_{\underline{\text{K}}}-1}\sum_{v=0}^{n_{\underline{\text{K}}}-1}\big(\underline{\text{K}}_{u,v} \cdot {\underline{\text{Z}}}_{p+u-c, q+v-c}\big)}_\text{Element-wise convolution}
				\bigg)
			\end{equation}
		\end{flushleft}
	\end{minipage}
	\vfill%	
	\end{frame}

	\begin{frame}[t]{Building a Field}
		\frametitle{Building a Field}
		\framesubtitle{2D Convolution}
		\hskip-0.75cm
		\begin{minipage}[t]{0.2\linewidth}\vspace{-0.5cm}
			\tiny\tableofcontents[currentsection,currentsubsection,hideothersubsections,subsectionstyle=show/shaded]
		\end{minipage}
		\hfill%
		\begin{minipage}[t]{0.86\linewidth}\vspace{-0.5cm}
			\begin{flushleft}
				Pseudo-code for 2D convolution with \textbf{bounds checking}
				\begin{algorithm}[H]
					\begin{algorithmic}[1]
						\FOR{$p=0$ to $m_{\underline{\text{Z}}}-1$}
							\FOR{$q=0$ to $n_{\underline{\text{Z}}}-1$}
								\FOR{$p=0$ to $m_{\underline{\text{K}}}-1$}
									\FOR{$v=0$ to $n_{\underline{\text{K}}}-1$}
									\IF{$p+u-c < m_{\underline{\text{Z}}}-1$ and $q+v-c < n_{\underline{\text{Z}}}-1$}
										\IF{$p+u-c \geq 0$ and $q+v-c \geq 0$}
										\STATE $\underline{\text{Z}}_{\text{new},p,q} = \big(\underline{\text{K}}_{u,v} \cdot {\underline{\text{Z}}}_{p+u-c, q+v-c}\big)$
									\ENDIF
									\ENDIF
									\ENDFOR
								\ENDFOR
							\ENDFOR
						\ENDFOR
						\STATE $\underline{\text{Z}}_{\text{new}} = \underline{\text{Z}}_{\text{new}} - \text{min}(\underline{\text{Z}}_{\text{new}})$
						\STATE $\underline{\text{Z}}_{\text{new}} = \underline{\text{Z}}_{\text{new}} \div \text{min}(\underline{\text{Z}}_{\text{new}})$
						\STATE $\underline{\text{Z}}_{\text{new}} = \underline{\text{Z}}_{\text{new}} \times \text{measurement max} - \text{measurement min}$
						\STATE $\underline{\text{Z}}_{\text{new}} = \underline{\text{Z}}_{\text{new}} + \text{measurement min}$
					\end{algorithmic}
					\label{alg:seq}
				\end{algorithm}
			\end{flushleft}
		\end{minipage}
		\vfill%
	\end{frame}
	
	\begin{frame}[t]{Building a Field}
		\frametitle{Building a Field}
		\framesubtitle{Gaussian Random Field (GRF)}
		\hskip-0.75cm
		\begin{minipage}[t]{0.2\linewidth}\vspace{-0.5cm}
			\tiny\tableofcontents[currentsection,currentsubsection,hideothersubsections,subsectionstyle=show/shaded]
		\end{minipage}
		\hfill%
		\begin{minipage}[t]{0.86\linewidth}\vspace{-0.5cm}
			\begin{figure}[t]
				\centering
				\captionsetup{width=0.9\textwidth}
				\includegraphics[width=0.6\textwidth]{../images/GRFafterConv.png}
				\caption{A GRF after convolving a 2D Gaussian kernel}
			\end{figure}
		\end{minipage}
		\vfill%
	\end{frame}

	%%%%%%%%%%%%%%%%%%%%%%%%%%%%%%%%%%%%%%%%%%%%%%%%%%%%%%%%%%%%%%%%%%%%%%%%%%%%%%%%%%%%%%%%%%%%%%%
	% Section: OrdinaryKriging
	%%%%%%%%%%%%%%%%%%%%%%%%%%%%%%%%%%%%%%%%%%%%%%%%%%%%%%%%%%%%%%%%%%%%%%%%%%%%%%%%%%%%%%%%%%%%%%%
	\section{Ordinary Kriging}
	\begin{frame}[t]{Ordinary Kriging}
		\frametitle{Ordinary Kriging}
		\hskip-0.75cm
		\begin{minipage}[t]{0.2\linewidth}\vspace{-0.5cm}
			\tiny\tableofcontents[currentsection,currentsubsection,hideothersubsections,subsectionstyle=show/shaded]
		\end{minipage}
		\hfill%
		\begin{minipage}[t]{0.86\linewidth}\vspace{-0.5cm}
			We use Ordinary Kriging, a form of Gaussian Process Regression (GPR) to estimate a field that is sparsely sampled
			\begin{itemize}
				\item Take $N$ measurements
				\item Form an Empirical/Experimental Semi-variogram $\underline{\text{E}}(h)$
				\item Fit a model of the variogram to the Empirical/Experimental Semi-variogram $\underline{\hat{\text{E}}}(h)$
				\item Form a covariance matrix, $\underline{\text{C}}$ based on $\hat{\underline{\text{E}}}(h)$
				\item Run the Kriging prediction to generate the field esimtate
			\end{itemize}
		\end{minipage}
		\vfill%
	\end{frame}

	\begin{frame}[t]{Ordinary Kriging}
		\frametitle{Ordinary Kriging}
		\framesubtitle{Take $N$-measurements}
		\hskip-0.75cm
		\begin{minipage}[t]{0.2\linewidth}\vspace{-0.5cm}
			\tiny\tableofcontents[currentsection,currentsubsection,hideothersubsections,subsectionstyle=show/shaded]
		\end{minipage}
		\hfill%
		\begin{minipage}[t]{0.86\linewidth}\vspace{-0.5cm}
			\begin{figure}[t]
				\centering
				\captionsetup{width=0.9\textwidth}
				\includegraphics[width=0.6\textwidth]{../images/MeasurementsOnField.png}
				\caption{Some measurements taken at various points on the field}
			\end{figure}
		\end{minipage}
		\vfill%
	\end{frame}

	\begin{frame}[t]{Ordinary Kriging}
		\frametitle{Ordinary Kriging}
		\framesubtitle{Form an Empirical Semi-variogram $\hat{\underline{\text{E}}}$}
		\hskip-0.75cm
		\begin{minipage}[t]{0.2\linewidth}\vspace{-0.5cm}
			\tiny\tableofcontents[currentsection,currentsubsection,hideothersubsections,subsectionstyle=show/shaded]
		\end{minipage}
		\hfill%
		\begin{minipage}[t]{0.86\linewidth}\vspace{-0.5cm}
			%%%% TODO: ADD FULL CITATION %%%%
			From Yonan 2019,
			\begin{align*}
				\forall \underline{\text{s}}_i,& \text{Z}(\underline{\text{s}}_i) \in \underline{\text{O}}:\\
				& \hat{\gamma}(h) \leftarrow \underline{\text{s}}_i, \text{Z}(\underline{\text{s}}_i)
			\end{align*}
			where 
			\begin{equation*}
				\underline{\text{s}}_i = 
				\left[\begin{array}{cc}
					\text{East}_i & \text{North}_i
				\end{array}\right]
			\end{equation*}
			and
			\begin{equation}\label{}
			\underline{\text{O}}
			=
			\left[\begin{array}{c}
				\text{Z}(\underline{\text{s}}_0)\\
				\text{Z}(\underline{\text{s}}_1)\\
				\vdots\\
				\text{Z}(\underline{\text{s}}_2)
				\end{array}\right]
			\end{equation}
			A Gaussian variogram model can be selected among other
			\begin{equation}\label{eq:GaussianModelVariogram}
				\gamma(h) = (s-n)\bigg(1-e^{-\big(\frac{h^2}{r^2a}\big)} + n\bigg)
			\end{equation}
			\textit{but} coefficients, $s, n, r^a$ must first be solved for.
		\end{minipage}
		\vfill%
	\end{frame}

	\begin{frame}[t]{Ordinary Kriging}
	\frametitle{Ordinary Kriging}
	\framesubtitle{Form an Empirical Semi-variogram $\hat{\underline{\text{E}}}$}
		\hskip-0.75cm
		\begin{minipage}[t]{0.2\linewidth}\vspace{-0.5cm}
			\tiny\tableofcontents[currentsection,currentsubsection,hideothersubsections,subsectionstyle=show/shaded]
		\end{minipage}
		\hfill%
		\begin{minipage}[t]{0.86\linewidth}\vspace{-0.5cm}
			The Empirical variogram assumes the following form,
			\begin{equation}
				\hat{\gamma}(h \pm \delta) := \frac{1}{2|N(h\pm\delta)|}\sum_{(i,j) \in N(h\pm\delta)}|\text{Z}(\underline{\text{s}}_i) - \text{Z}(\underline{\text{s}}_j)|^2
			\end{equation}
			and
			\begin{equation}
				 \hat{\underline{\text{E}}} = 
				 \left[\begin{array}{c}
					 \hat{\gamma}(\delta)\\
					 \hat{\gamma}(2*\delta)\\
					 \vdots\\
					 \hat{\gamma}(N*\delta)
				 \end{array}\right]
			\end{equation}
		\end{minipage}
		\vfill%
	\end{frame}

	\subsection{Fitting the Variogram}
	\begin{frame}[t]{Ordinary Kriging}
	\frametitle{Ordinary Kriging}
	\framesubtitle{Fitting $\underline{\text{E}}$ to $\hat{\underline{\text{E}}}$}
		\hskip-0.75cm
		\begin{minipage}[t]{0.2\linewidth}\vspace{-0.5cm}
			\tiny\tableofcontents[currentsection,currentsubsection,hideothersubsections,subsectionstyle=show/shaded]
		\end{minipage}
		\hfill%
		\begin{minipage}[t]{0.86\linewidth}\vspace{-0.5cm}
			Least-squares (LS) \textit{can} be used for fitting. The lag-vector is
			\begin{equation}
				\underline{\text{h}} = \left[\begin{array}{cccc}
					d_{xy} & 2d_{xy} & \dots & Nd_{xy}
				\end{array}\right]^\text{T}
			\end{equation}
			Then, the following $N \times 3$ matrix may be formed with a 3rd-order Taylor approximation
			\begin{equation}
				\underline{\text{H}}=
				\left[\begin{array}{ccc}
				\underline{\text{h}}^2 & -\underline{\text{h}}^4 & \underline{1}
				\end{array}\right]
			\end{equation}
			Now, apply LS
			\begin{equation}
				\underbrace{\hat{\underline{\text{x}}}}_{3 \times 1} = (\underbrace{\underline{\text{H}}^T\underline{\text{H}}}_{3 \times 3})^{-1}\underline{\text{H}}^\text{T}\hat{\underline{\text{E}}}
			\end{equation}
			Now the \textbf{sill}, the \textbf{range}, and the \textbf{nugget} coefficients can be solved for because we have 3 equations and 3 unknowns (assuming \textit{good} measurements)
		\end{minipage}
		\vfill%
	\end{frame}

	\begin{frame}[t]{Ordinary Kriging}
		\frametitle{Ordinary Kriging}
		\framesubtitle{Fitting $\underline{\text{E}}$ to $\hat{\underline{\text{E}}}$}
		\hskip-0.75cm
		\begin{minipage}[t]{0.2\linewidth}\vspace{-0.5cm}
			\tiny\tableofcontents[currentsection,currentsubsection,hideothersubsections,subsectionstyle=show/shaded]
		\end{minipage}
		\hfill%
		\begin{minipage}[t]{0.86\linewidth}\vspace{-0.5cm}
			The system of equations is as follows
			\begin{align}
				s &= (r^2a)\hat{\underline{\text{x}}}_0 + n\\
				n &= \hat{\underline{\text{x}}}_2 + n\\
				(r^2a) &= \frac{\hat{\underline{\text{x}}}_0}{2\hat{\underline{\text{x}}}_1}
			\end{align}
			Now, equation~\ref{eq:GaussianModelVariogram},
			\begin{equation*}
			\gamma(h) = (s-n)\bigg(1-e^{-\big(\frac{h^2}{r^2a}\big)} + n\bigg)
			\end{equation*}
			can be used in forming the covariance matrix, $\underline{\text{C}}$
		\end{minipage}
		\vfill%
	\end{frame}

	\begin{frame}[t]{Ordinary Kriging}
		\frametitle{Ordinary Kriging}
		\framesubtitle{Fitting $\underline{\text{E}}$ to $\hat{\underline{\text{E}}}$}
		\hskip-0.75cm
		\begin{minipage}[t]{0.2\linewidth}\vspace{-0.5cm}
			\tiny\tableofcontents[currentsection,currentsubsection,hideothersubsections,subsectionstyle=show/shaded]
		\end{minipage}
		\hfill%
		\begin{minipage}[t]{0.86\linewidth}\vspace{-0.5cm}
			\begin{figure}[t]
				\centering
				\captionsetup{width=0.9\textwidth}
				\includegraphics[width=\textwidth]{../images/fittingE.png}
				\caption{A Gaussian variogram model fitted to an Empirical variogram}
			\end{figure}
		\end{minipage}
		\vfill%
	\end{frame}
	
	\subsection{The Covariance Matrix}
	\begin{frame}[t]{Ordinary Kriging}
		\frametitle{Ordinary Kriging}
		\framesubtitle{The Covariance Matrix}
		\hskip-0.75cm
		\begin{minipage}[t]{0.2\linewidth}\vspace{-0.5cm}
			\tiny\tableofcontents[currentsection,currentsubsection,hideothersubsections,subsectionstyle=show/shaded]
		\end{minipage}
		\hfill%
		\begin{minipage}[t]{0.86\linewidth}\vspace{-0.5cm}
			The covariance matrix $\underline{\text{C}}$
			\begin{equation}
				\sum_{i=0}^{N}
					\sum_{j=0}^{N}
						\text{C}_{i,j}=\gamma(h) \: | \: h=||\underline{\text{s}}_i - \underline{\text{s}}_j||
			\end{equation}
		\end{minipage}
		\vfill%
	\end{frame}

	\subsection{Predicting the Field}
	\begin{frame}[t]{Ordinary Kriging}
		\frametitle{Ordinary Kriging}
		\framesubtitle{Predicting the Field}
		\hskip-0.75cm
		\begin{minipage}[t]{0.2\linewidth}\vspace{-0.5cm}
			\tiny\tableofcontents[currentsection,currentsubsection,hideothersubsections,subsectionstyle=show/shaded]
		\end{minipage}
		\hfill%
		\begin{minipage}[t]{0.86\linewidth}\vspace{-0.5cm}
			The estimate is formed as follows,
			\begin{align*}
				\forall \underline{p}_i &\in \underline{\text{Z}}_\text{Field}\\
				& \underline{\text{d}}_i = 
				\left[\begin{array}{cccc}
					\gamma(||\underline{\text{s}}_0 - \underline{\text{p}}_i||) & \dots & \gamma(||\underline{\text{s}}_{N-1} - \underline{\text{p}}_i||) & \underline{1}
				\end{array}\right]^\text{T}\\
				& \left[\begin{array}{c}
				\lambda_{\underline{\text{p}}_i}\\
				\eta_{\underline{\text{p}}_i}
				\end{array}\right]
				=
				\left[\begin{array}{cc}
				\underline{\text{C}}^{-1} & \underline{1}\\
				\underline{1}^\text{T} & 0
				\end{array}\right]
				\left[\begin{array}{c}
				\underline{\text{d}}_\underline{\text{p}_i}\\
				1
				\end{array}\right]\\
				& \hat{\underline{\text{Z}}}(\underline{\text{p}_i}) = \left[\begin{array}{ccc}
				\underline{\text{Z}}(\underline{\text{s}}_0) & \dots & \underline{\text{Z}}(\underline{\text{s}}_{N-1})
				\end{array}\right]^\text{T}
				\lambda_{\underline{\text{p}}_i}\\
				&\text{var}(\hat{\underline{\text{Z}}}(\underline{\text{p}_i})) = 
				\left[\begin{array}{cc}
				\underline{\text{d}}_{\underline{\text{p}}_i} & 1
				\end{array}\right]
				\left[\begin{array}{c}
				\lambda_{\underline{\text{p}}_i}\\
				\eta_{\underline{\text{p}}_i}
				\end{array}\right]
			\end{align*}
		\end{minipage}
		\vfill%
	\end{frame}

	\begin{frame}[t]{Ordinary Kriging}
		\frametitle{Ordinary Kriging}
		\framesubtitle{Predicting the Field}
		\hskip-0.75cm
		\begin{minipage}[t]{0.2\linewidth}\vspace{-0.5cm}
			\tiny\tableofcontents[currentsection,currentsubsection,hideothersubsections,subsectionstyle=show/shaded]
		\end{minipage}
		\hfill%
		\begin{minipage}[t]{0.86\linewidth}\vspace{-0.5cm}
			\begin{figure}[t]
				\centering
				\captionsetup{width=\textwidth}
				\includegraphics[width=\textwidth]{../images/estimatedField.png}
				\caption{The estimated field (left) and the true-field with measurements (right)}
			\end{figure}
		\end{minipage}
		\vfill%
	\end{frame}
	%%%%%%%%%%%%%%%%%%%%%%%%%%%%%%%%%%%%%%%%%%%%%%%%%%%%%%%%%%%%%%%%%%%%%%%%%%%%%%%%%%%%%%%%%%%%%%%
	% Section: References
	%%%%%%%%%%%%%%%%%%%%%%%%%%%%%%%%%%%%%%%%%%%%%%%%%%%%%%%%%%%%%%%%%%%%%%%%%%%%%%%%%%%%%%%%%%%%%%%
%	\section{References}
%	\begin{frame}[allowframebreaks]{Bibliography}
%		\frametitle{References}
%		\bibliographystyle{ieeetr}
%		\bibliography{../../somefolder/some_bib_file}
%		
%	\end{frame}


	%%%%%%%%%%%%%%%%%%%%%%%%%%%%%%%%%%%%%%%%%%%%%%%%%%%%%%%%%%%%%%%%%%%%%%%%%%%%%%%%%%%%%%%%%%%%%%%
	
\end{document}