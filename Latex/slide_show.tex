\documentclass[professionalfont,10pt]{beamer}
\usetheme{Pavlo}				% My custom theme

\usepackage{newtxtext} % For Times New Roman font
\usepackage[utf8]{inputenc}
\usepackage[T1]{fontenc}
%Information to be included in the title page:
\title{Introduction to Gaussian Process Regression as Ordinary Kriging}
\subtitle{A Brief Overview Presentation}
\author{Pavlo Vlastos}
%\date{August 25th, 2020}
\institute{Autonomous Systems Laboratory \\ The University of California at Santa Cruz}
\titlegraphic{\includegraphics[width=2.5cm]{./../images/ucsc-seal.jpg}}
\usepackage{caption}
\usepackage[font=footnotesize]{caption}
\captionsetup{font=footnotesize}
\usepackage{subcaption}
\usepackage{multicol}
\usepackage{bibentry}
\usepackage{hyperref}
\usepackage{lgreek}
\usepackage{algorithm2e,algorithmic,float}
\setbeamertemplate{bibliography item}[text]
\AtBeginSection

\begin{document}
	
	\begin{frame}
		\titlepage
	\end{frame}

	\begin{frame}{Table of Contents}
		\frametitle{Table of Contents}
		\begin{minipage}[t]{0.2\linewidth}\vspace{-0.5cm}
			\vfill%
		\end{minipage}
			\hfill%
		\begin{minipage}[t]{0.86\linewidth}\vspace{-0.5cm}
			%			\begin{multicols}{2} % USE IF THE TABLE OF CONTENTS IS TOO LONG
				\tableofcontents
			%			\end{multicols}
		\end{minipage}
		\vfill%
	\end{frame}
	
	%%%%%%%%%%%%%%%%%%%%%%%%%%%%%%%%%%%%%%%%%%%%%%%%%%%%%%%%%%%%%%%%%%%%%%%%%%%%%%%%%%%%%%%%%%%%%%%
	% Section: Introduction
	%%%%%%%%%%%%%%%%%%%%%%%%%%%%%%%%%%%%%%%%%%%%%%%%%%%%%%%%%%%%%%%%%%%%%%%%%%%%%%%%%%%%%%%%%%%%%%%
	\section{Introduction}
	\begin{frame}[t]{Introduction}
		\frametitle{Introduction}
		\hskip-0.75cm
		\begin{minipage}[t]{0.2\linewidth}\vspace{-0.5cm}
			\tiny\tableofcontents[currentsection,currentsubsection,hideothersubsections,subsectionstyle=show/shaded]
		\end{minipage}
		\hfill%
		\begin{minipage}[t]{0.86\linewidth}\vspace{-0.5cm}
			
		\end{minipage}
	\vfill%
	\end{frame}

	%%%%%%%%%%%%%%%%%%%%%%%%%%%%%%%%%%%%%%%%%%%%%%%%%%%%%%%%%%%%%%%%%%%%%%%%%%%%%%%%%%%%%%%%%%%%%%%
	% Section: Building a Field
	%%%%%%%%%%%%%%%%%%%%%%%%%%%%%%%%%%%%%%%%%%%%%%%%%%%%%%%%%%%%%%%%%%%%%%%%%%%%%%%%%%%%%%%%%%%%%%%
	\section{Building a Field}
	\begin{frame}[t]{Building a Field}
		\frametitle{Building a Field}
		\framesubtitle{Gaussian Random Field (GRF)}
		\hskip-0.75cm
		\begin{minipage}[t]{0.2\linewidth}\vspace{-0.5cm}
			\tiny\tableofcontents[currentsection,currentsubsection,hideothersubsections,subsectionstyle=show/shaded]
		\end{minipage}
		\hfill%
		\begin{minipage}[t]{0.86\linewidth}\vspace{-0.5cm}
			\begin{figure}[t]
				\centering
				\captionsetup{width=0.9\textwidth}
				\includegraphics[width=0.6\textwidth]{../images/GRFbeforeConv.png}
				\caption{A Gaussian Random Field (GRF) with values separated by a desired spacing parameter, or resolution}
				\label{CFandEncoder}
			\end{figure}
		\end{minipage}
		\vfill%
	\end{frame}

	\begin{frame}[t]{Building a Field}
		\frametitle{Building a Field}
		\framesubtitle{Gaussian Random Field (GRF)}
		\hskip-0.75cm
		\begin{minipage}[t]{0.2\linewidth}\vspace{-0.5cm}
			\tiny\tableofcontents[currentsection,currentsubsection,hideothersubsections,subsectionstyle=show/shaded]
		\end{minipage}
		\hfill%
		\begin{minipage}[t]{0.86\linewidth}\vspace{-0.5cm}
			\begin{flushleft}
				$d_{xy}$ : smallest distance separating any two adjacent random values 
				
				$\underline{\text{x}}$ : $1 \times n$ Vector of x-coordinates, $\left[\begin{array}{cccc}x_0 & x_1 & \dots & x_{n-1}\end{array}\right]$,\\
				$ x_i~=~jd_{xy}, j~\in~\mathbb{Z}$
				
				$\underline{\text{y}}$ : $m \times 1$ Vector of y-coordinates, $\left[\begin{array}{cccc}y_0 & y_1 & \dots & y_{n-1}\end{array}\right]$,\\
				$ y_i~=~jd_{xy}, j~\in~\mathbb{Z}$

				
				$\underline{\text{Z}}$ : $m \times n$ Matrix of z-axis field values 
				
				$\underline{\text{K}}$ : $k \times k$ Matrix representing the kernel for convolution
	
			\end{flushleft}
		\end{minipage}
		\vfill%
	\end{frame}
	
	\subsection{Gaussian Kernel}
	\begin{frame}[t]{Building a Field}
		\frametitle{Building a Field}
		\framesubtitle{Gaussian Kernel}
		\hskip-0.75cm
		\begin{minipage}[t]{0.2\linewidth}\vspace{-0.5cm}
			\tiny\tableofcontents[currentsection,currentsubsection,hideothersubsections,subsectionstyle=show/shaded]
		\end{minipage}
		\hfill%
		\begin{minipage}[t]{0.86\linewidth}\vspace{-0.5cm}
			\begin{flushleft}
				\begin{itemize}
					\item A kernel is required to smooth the GRF to better describe some phenomenon. A 2D Gaussian function can be chosen for a Gaussian kernel,
					\begin{equation}
						g(x, y) = \frac{1}{2\pi\sigma^2}e^{-\Big(\frac{(x-\mu_x)^2 + (y-\mu_y)^2}{2\sigma^2}\Big)}
					\end{equation}
					where $\mu_x$ and $\mu_y$ are the respective means of the distributions, but usually are zero, and $\sigma$ is the standard deviation. \\
					\item To form the Gaussian kernel:
					\begin{equation}
					\underline{\text{K}} = \frac{1}{\sum_{i=0}^{k-1}\sum_{j=0}^{k-1} \text{K}_{i, j}}\Bigg(\sum_{i=0}^{k-1}\sum_{j=0}^{k-1} \text{K}_{i, j} = g(j, i)\Bigg)
					\end{equation}
					where $\text{K}_{i,j}$ is an element of $\underline{\text{K}}$
				\end{itemize}
			\end{flushleft}
		\end{minipage}
		\vfill%	
	\end{frame}

	\begin{frame}[t]{Building a Field}
		\frametitle{Building a Field}
		\framesubtitle{Gaussian Kernel}
		\hskip-0.75cm
		\begin{minipage}[t]{0.2\linewidth}\vspace{-0.5cm}
			\tiny\tableofcontents[currentsection,currentsubsection,hideothersubsections,subsectionstyle=show/shaded]
		\end{minipage}
		\hfill%
		\begin{minipage}[t]{0.86\linewidth}\vspace{-0.5cm}
			\begin{flushleft}
				A pseudo code implementation for the Gaussian kernel may assume the following form:
				\begin{algorithm}[H]
					\begin{algorithmic}[1]
						\STATE \text{sum} = 0
						\FOR{$i=0$ to $k-1$}
						\FOR{$j=0$ to $k-1$}
						\STATE $\underline{K}_{i, j} = g(j, i)$
						\STATE $\text{sum} += \underline{K}_{i, j}$
						\ENDFOR
						\ENDFOR
						\STATE $\underline{K}_{i, j} /= \text{sum}$
					\end{algorithmic}
					\label{alg:seq}
				\end{algorithm}
			\end{flushleft}
		\end{minipage}
		\vfill%
	\end{frame}
	
	\begin{frame}[t]{Building a Field}
		\frametitle{Building a Field}
		\framesubtitle{Gaussian Kernel}
		\hskip-0.75cm
		\begin{minipage}[t]{0.2\linewidth}\vspace{-0.5cm}
			\tiny\tableofcontents[currentsection,currentsubsection,hideothersubsections,subsectionstyle=show/shaded]
		\end{minipage}
		\hfill%
		\begin{minipage}[t]{0.86\linewidth}\vspace{-0.5cm}
			\begin{figure}[t]
				\centering
				\captionsetup{width=0.9\textwidth}
				\includegraphics[width=0.7\textwidth]{../images/GaussianKernel.png}
				\caption{A 2D Gaussian kernel, $\mu_x = \mu_y = 0, \sigma = 8$}
				\label{CFandEncoder}
			\end{figure}
		\end{minipage}
		\vfill%
	\end{frame}
	
	\subsection{2D Convolution}
	\begin{frame}[t]{Building a Field}
	\frametitle{Building a Field}
	\framesubtitle{2D Convolution}
	\hskip-0.75cm
	\begin{minipage}[t]{0.2\linewidth}\vspace{-0.5cm}
		\tiny\tableofcontents[currentsection,currentsubsection,hideothersubsections,subsectionstyle=show/shaded]
	\end{minipage}
	\hfill%
	\begin{minipage}[t]{0.86\linewidth}\vspace{-0.5cm}
		\begin{flushleft}
			The Gaussian kernel can be convolved with the GRF
			\begin{equation}
				\underline{\text{Z}}_{\text{new}} = \sum_{p=0}^{m_{\underline{\text{Z}}}-1}\sum_{q=0}^{n_{\underline{\text{Z}}}-1}\bigg(
				\underbrace{\underline{\text{Z}}_{\text{new},p,q} = \sum_{u=0}^{m_{\underline{\text{K}}}-1}\sum_{v=0}^{n_{\underline{\text{K}}}-1}\big(\underline{\text{K}}_{u,v} \cdot {\underline{\text{Z}}}_{p+u-c, q+v-c}\big)}_\text{Element-wise convolution}
				\bigg)
			\end{equation}
		\end{flushleft}
	\end{minipage}
	\vfill%	
	\end{frame}

	\begin{frame}[t]{Building a Field}
		\frametitle{Building a Field}
		\framesubtitle{2D Convolution}
		\hskip-0.75cm
		\begin{minipage}[t]{0.2\linewidth}\vspace{-0.5cm}
			\tiny\tableofcontents[currentsection,currentsubsection,hideothersubsections,subsectionstyle=show/shaded]
		\end{minipage}
		\hfill%
		\begin{minipage}[t]{0.86\linewidth}\vspace{-0.5cm}
			\begin{flushleft}
				Pseudo-code for 2D convolution with \textbf{bounds checking}
				\begin{algorithm}[H]
					\begin{algorithmic}[1]
						\FOR{$p=0$ to $m_{\underline{\text{Z}}}-1$}
							\FOR{$q=0$ to $n_{\underline{\text{Z}}}-1$}
								\FOR{$p=0$ to $m_{\underline{\text{K}}}-1$}
									\FOR{$v=0$ to $n_{\underline{\text{K}}}-1$}
									\IF{$p+u-c < m_{\underline{\text{Z}}}-1$ and $q+v-c < n_{\underline{\text{Z}}}-1$}
										\IF{$p+u-c \geq 0$ and $q+v-c \geq 0$}
										\STATE $\underline{\text{Z}}_{\text{new},p,q} = \big(\underline{\text{K}}_{u,v} \cdot {\underline{\text{Z}}}_{p+u-c, q+v-c}\big)$
									\ENDIF
									\ENDIF
									\ENDFOR
								\ENDFOR
							\ENDFOR
						\ENDFOR
						\STATE $\underline{\text{Z}}_{\text{new}} = \underline{\text{Z}}_{\text{new}} - \text{min}(\underline{\text{Z}}_{\text{new}})$
						\STATE $\underline{\text{Z}}_{\text{new}} = \underline{\text{Z}}_{\text{new}} \div \text{min}(\underline{\text{Z}}_{\text{new}})$
						\STATE $\underline{\text{Z}}_{\text{new}} = \underline{\text{Z}}_{\text{new}} \times \text{measurement max} - \text{measurement min}$
						\STATE $\underline{\text{Z}}_{\text{new}} = \underline{\text{Z}}_{\text{new}} + \text{measurement min}$
					\end{algorithmic}
					\label{alg:seq}
				\end{algorithm}
			\end{flushleft}
		\end{minipage}
		\vfill%
	\end{frame}
	
	\begin{frame}[t]{Building a Field}
		\frametitle{Building a Field}
		\framesubtitle{Gaussian Random Field (GRF)}
		\hskip-0.75cm
		\begin{minipage}[t]{0.2\linewidth}\vspace{-0.5cm}
			\tiny\tableofcontents[currentsection,currentsubsection,hideothersubsections,subsectionstyle=show/shaded]
		\end{minipage}
		\hfill%
		\begin{minipage}[t]{0.86\linewidth}\vspace{-0.5cm}
			\begin{figure}[t]
				\centering
				\captionsetup{width=0.9\textwidth}
				\includegraphics[width=0.6\textwidth]{../images/GRFafterConv.png}
				\caption{A GRF after convolving a 2D Gaussian kernel}
				\label{CFandEncoder}
			\end{figure}
		\end{minipage}
		\vfill%
	\end{frame}

	%%%%%%%%%%%%%%%%%%%%%%%%%%%%%%%%%%%%%%%%%%%%%%%%%%%%%%%%%%%%%%%%%%%%%%%%%%%%%%%%%%%%%%%%%%%%%%%
	% Section: Kriging
	%%%%%%%%%%%%%%%%%%%%%%%%%%%%%%%%%%%%%%%%%%%%%%%%%%%%%%%%%%%%%%%%%%%%%%%%%%%%%%%%%%%%%%%%%%%%%%%
	\section{Kriging}
	\begin{frame}[t]{Kriging}
		\frametitle{Kriging}
		\hskip-0.75cm
		\begin{minipage}[t]{0.2\linewidth}\vspace{-0.5cm}
			\tiny\tableofcontents[currentsection,currentsubsection,hideothersubsections,subsectionstyle=show/shaded]
		\end{minipage}
		\hfill%
		\begin{minipage}[t]{0.86\linewidth}\vspace{-0.5cm}
			\begin{itemize}
				
			\end{itemize}
		\end{minipage}
		\vfill%
	\end{frame}

	%%%%%%%%%%%%%%%%%%%%%%%%%%%%%%%%%%%%%%%%%%%%%%%%%%%%%%%%%%%%%%%%%%%%%%%%%%%%%%%%%%%%%%%%%%%%%%%
	% Section: References
	%%%%%%%%%%%%%%%%%%%%%%%%%%%%%%%%%%%%%%%%%%%%%%%%%%%%%%%%%%%%%%%%%%%%%%%%%%%%%%%%%%%%%%%%%%%%%%%
%	\section{References}
%	\begin{frame}[allowframebreaks]{Bibliography}
%		\frametitle{References}
%		\bibliographystyle{ieeetr}
%		\bibliography{../../somefolder/some_bib_file}
%		
%	\end{frame}


	%%%%%%%%%%%%%%%%%%%%%%%%%%%%%%%%%%%%%%%%%%%%%%%%%%%%%%%%%%%%%%%%%%%%%%%%%%%%%%%%%%%%%%%%%%%%%%%
	
\end{document}